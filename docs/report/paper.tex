%"runningheads" enables:
%  - page number on page 2 onwards
%  - title/authors on even/odd pages
%This is good for other readers to enable proper archiving among other papers and pointing to content.
%Even if the title page states the title, when printed and stored in a folder, when blindly opening the folder, one could hit not the title page, but an arbitrary page. Therefore, it is good to have title printed on the pages, too.
\documentclass[runningheads,a4paper]{llncs}

%Even though `american`, `english` and `USenglish` are synonyms for babel package (according to https://tex.stackexchange.com/questions/12775/babel-english-american-usenglish), the llncs document class is prepared to avoid the overriding of certain names (such as "Abstract." -> "Abstract" or "Fig." -> "Figure") when using `english`, but not when using the other 2.
\usepackage[english]{babel}

%better font, similar to the default springer font
%cfr-lm is preferred over lmodern. Reasoning at http://tex.stackexchange.com/a/247543/9075
\usepackage[%
rm={oldstyle=false,proportional=true},%
sf={oldstyle=false,proportional=true},%
tt={oldstyle=false,proportional=true,variable=true},%
qt=false%
]{cfr-lm}
%
%if more space is needed, exchange cfr-lm by mathptmx
%\usepackage{mathptmx}

\usepackage{graphicx}

%extended enumerate, such as \begin{compactenum}
\usepackage{paralist}

%put figures inside a text
%\usepackage{picins}
%use
%\piccaptioninside
%\piccaption{...}
%\parpic[r]{\includegraphics ...}
%Text...

%Sorts the citations in the brackets
%\usepackage{cite}

\usepackage[T1]{fontenc}

%for demonstration purposes only
\usepackage[math]{blindtext}

%for easy quotations: \enquote{text}
\usepackage{csquotes}

%enable margin kerning
\usepackage{microtype}

%tweak \url{...}
\usepackage{url}
%nicer // - solution by http://tex.stackexchange.com/a/98470/9075
\makeatletter
\def\Url@twoslashes{\mathchar`\/\@ifnextchar/{\kern-.2em}{}}
\g@addto@macro\UrlSpecials{\do\/{\Url@twoslashes}}
\makeatother
\urlstyle{same}
%improve wrapping of URLs - hint by http://tex.stackexchange.com/a/10419/9075
\makeatletter
\g@addto@macro{\UrlBreaks}{\UrlOrds}
\makeatother

%diagonal lines in a table - http://tex.stackexchange.com/questions/17745/diagonal-lines-in-table-cell
%slashbox is not available in texlive (due to licensing) and also gives bad results. This, we use diagbox
%\usepackage{diagbox}

%required for pdfcomment later
\usepackage{xcolor}

% new packages BEFORE hyperref
% See also http://tex.stackexchange.com/questions/1863/which-packages-should-be-loaded-after-hyperref-instead-of-before

%enable hyperref without colors and without bookmarks
\usepackage[
%pdfauthor={},
%pdfsubject={},
%pdftitle={},
%pdfkeywords={},
bookmarks=false,
breaklinks=true,
colorlinks=true,
linkcolor=black,
citecolor=black,
urlcolor=black,
%pdfstartpage=19,
pdfpagelayout=SinglePage,
pdfstartview=Fit
]{hyperref}
%enables correct jumping to figures when referencing
\usepackage[all]{hypcap}

%enable nice comments
\usepackage{pdfcomment}
\newcommand{\commentontext}[2]{\colorbox{yellow!60}{#1}\pdfcomment[color={0.234 0.867 0.211},hoffset=-6pt,voffset=10pt,opacity=0.5]{#2}}
\newcommand{\commentatside}[1]{\pdfcomment[color={0.045 0.278 0.643},icon=Note]{#1}}

%compatibality with TODO package
\newcommand{\todo}[1]{\commentatside{#1}}

%enable \cref{...} and \Cref{...} instead of \ref: Type of reference included in the link
\usepackage[capitalise,nameinlink]{cleveref}
%Nice formats for \cref
\crefname{section}{Sect.}{Sect.}
\Crefname{section}{Section}{Sections}

\usepackage{xspace}
%\newcommand{\eg}{e.\,g.\xspace}
%\newcommand{\ie}{i.\,e.\xspace}
\newcommand{\eg}{e.\,g.,\ }
\newcommand{\ie}{i.\,e.,\ }

% correct bad hyphenation here
\hyphenation{op-tical net-works semi-conduc-tor}

\begin{document}

\input glyphtounicode.tex
\pdfgentounicode=1

\title{\texttt{timeline}: A Time Series Visualization Platform}
\author{Manuel Haid \and Thomas Mauerhofer \and Matthias W\"olbitsch}
\institute{Knowledge Technologies Institute}

\maketitle

\begin{abstract}
Abstract goes here
\end{abstract}

\keywords{Time Series, Visualization, Forecasting}

\section{Introduction}\label{sec:intro}
%Motivation and goals (which problem you are solving for the chosen data)

In today's society data plays a import role in many different ways.
For instance, businesses are using large amount of data to predict consumer behavior to satisfy their demand.
Often time plays a essential role in these real-world data sets and applications. 
The following list contains three representative examples in no particular order:

\begin{enumerate}
 \item The consumption of electricity does not only depend on the time of the day, but also on the day of the week and other seasonal effects.
 It is important to model the future consumption to avoid failures of large portions or the complete power grid.
 \item In the area of weather forecasting time is of the essence. 
 Many applications (\eg shipping) require an precise and accurate forecasts to operate on minimal risk.
 \item In the finance market many decisions are made within small time spans and are based on the prediction of complex models.
 Incorrect or late data can lead to significant financial losses.
\end{enumerate}


This kind of data sets with an temporal importance are usually referred to as time series (\ie two- or higher-dimensional data with time as one dimension). 
A more formal definition \cite{Cortez2012} of a time series \(\{Y\}_t\) is a time ordered collection of observations \((y_1, y_2, \ldots, y_n)\).
This implies two trivial consequences. 
First, the samples in the time series are ordered, and second, privious observations may influcence future values but never the other way around.
A element of the time series recorded at time or period \(t\), denoted as \(y_t\), can either be a scalar value or a vector.
Time series of scalars are denoted as \emph{univariate} times series, whereas a time series of higher dimensional data is called a \emph{multivariate} time series.
For this project we only focused on univariate time series data and refer to to them simply as time series in the context of this paper.

When dealing with time series usually one of the first steps in the analysis process is the visualization of the data.
The graphical representation can already give some great insights in properties and characteristics of the data.
For example, a simple visual inspection can show patterns, trends or seasonal effects without the use of advanced statistical methods. 
These properties are important since they have to be considered in the modeling of the time series.
One example for a statistical forecasting model that requires knowledge of the seasonal and stationary properties of a data set is Box--Jenkins \cite{Box1976} model.

The goal of our project is to create a web-based platform that allows the user to perform this initial visual inspection of the time series in a convenient but sophisticated way.
This includes, of course, the possibility to create simple time series plots (\ie visualizations that plot the data against the time) but also more advanced plots such as the auto-correlation function (ACF) plot.
Another goal is the support of a quick and responsive visualization of forecasts on the a data set. 
This means that is should be possible to quickly inspect the deviation of the prediction from the actual test data including a confidence band.

The rest of the paper is structured as follows. 
\Cref{sec:solution} contains a detailed description of our platform, used algorithms and overall design.
\Cref{sec:conclusion} contains the conclusion of the project and a discussion of possible future work and improvements.


\section{Solution}\label{sec:solution}
%Description of your solution: methodology, algorithms, design, use case

\section{Conclusion and Outlook}\label{sec:conclusion}
% Discussion and outlook: what worked well, what could be improved


\bibliographystyle{splncs03}
\bibliography{paper}

All links were last followed on October 5, 2014. % TODO: change if there are any web ressources

\end{document}
