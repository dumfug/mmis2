\documentclass[11pt, a4paper]{article}

\usepackage{amssymb, amsmath}
\usepackage[utf8]{inputenc}
\usepackage{fullpage}
\usepackage{graphicx}
\usepackage{listings}
\usepackage{hyperref}
\usepackage{parskip}
\usepackage{gensymb}


\author{Manuel Haid \and Thomas Mauerhofer \and Matthias Wölbitsch}
\title{\texttt{timeline} Documentation}

% content:
% it's about „Technical and code“ documentation (code)
% Requirements
% Libraries / frameworks used
% Which step/code does what in detail
% preferably with flow charts
% One should be able to extend or adapt your code easily after reading your documentation

\begin{document}

\maketitle
\tableofcontents

\section{Introduction} \label{sec:intoduction}


\section{Requirements} \label{sec:requirments}

\begin{itemize}
 \item \textgreater= Python 3.5
\end{itemize}
 

\section{Dependencies} \label{sec:dependencies}

This section contains a description of all used external dependencies (e.g. libraries) and why they were used in this project.
The python dependencies can be installed using \texttt{pip}, the Python package manager, using the \texttt{requirements.txt} file. 
It is recommended to use a virtual environments when developing for this project to avoid conflicts with other system-wide installed versions of the same dependencies.


\subsection{Back-end (Python)}

\begin{description}
 \item[Flask] 
 Flask is a microframework (simple but extendable) for web development. 
 It allows RESTful request dispatching and is well documented. 
 In this project Flask is used for the rendering of the front-end using its template engine and providing an API for the supported visualizations.
 
 \item[Pandas] 
 Pandas is a widely used data analysis library. 
 It provides data structures to efficiently store and query in-memory data. 
 All temporal data (i.e. all time series) in this project is stored in Pandas DataFrame or Series objects. 
 
 \item[NumPy] 
 NumPy is a powerful package for scientific computing. 
 It provides efficient data structures and algorithms to operate on them. 
 However, in this project this library is only used to sample random data for the example data sets and is therefore not very important. 
\end{description}


\subsection{Front-end (HTML5)}

\begin{description}
 \item[jQuery]
 \item[MetricsGraphics.js]
 \item[Bootstrap]
\end{description}


\end{document}
